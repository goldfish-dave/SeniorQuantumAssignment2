\documentclass[10pt,a4paper]{article}
\usepackage{amssymb}
\usepackage{fullpage}
\usepackage{amsmath}

\title{Senior Quantum 2011 Assignment 2}
\date{}
\author{D. G. Wilcox \\
		309248035}

\begin{document}
\maketitle
\section*{Question 1}
\begin{itemize}
	\item[(a)] If $A$ is the atomic transition rate from the excited state to theground state, the population of the excited state $N$ is described by the equation:
		\begin{equation*}
			\frac{dN}{dt} = - A N
		\end{equation*}
	which we can solve to get:
		\begin{align*}
			N(t) &= N_{0}e^{-At} \\
			\Rightarrow \tau &= \frac{1}{A} \\
		\end{align*}
	
	\item[(b)] The rate of transition is $A$, therefore $\Delta t = \tau$, giving:
		\begin{align*}
			\Delta E &\geq \frac{\hbar}{2 \Delta t} \\
			&\geq 3.94 \times 10^{-27} \\
		\end{align*}
	where $\Delta E$ is measured in Joules.

	\item[(c)] We can find a relation between $\Delta \lambda$ and $\Delta E$ via the following:
		\begin{align*}
			E &= \frac{hc}{\Lambda} \\
			\Delta E &= E_{1} - E_{2} = hc [\frac{1}{\lambda_{1}} - \frac{1}{\lambda_{2}}] \\
			&= hc [\frac{\lambda_{2} - \lambda_{1}}{\lambda_{1}\lambda_{2}}] \\
			&= hc [\frac{\Delta \lambda}{\lambda^{2}}] \\
			\Rightarrow \Delta \lambda &= \Delta E \frac{\lambda^{2}}{hc} \\
			&= 4.02 \times 10^{-15} \\
		\end{align*}
	where $\Delta \lambda$ is in metres.

	\item[(d)] A forbidden transition has a much smaller transition probability. This means $\Delta t$ is larger and so $\Delta E$ is smaller.
\end{itemize}


\section*{Question 2}

\section*{Question 3}
\section*{Question 4}
\end{document}
